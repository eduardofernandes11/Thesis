\begin{table}[ht]
    \centering
    \begin{tabular}{p{3cm}p{10cm}p{1cm}}
    \hline
    \textbf{Dataset} & \textbf{Description} & \textbf{Used by} \\
    \hline
    TREC Public Coprus & Collection of email
    messages collected between April 8 and July 6, 2007. There are about 50,000 deceptive emails and about 25,000 legitimate. & \cite{rabbi2023phishy} \\
    \hline
    Ling & The Ling-Spam dataset is a collection of 2,893 spam and non-spam messages curated from the Linguist List.  & \cite{rabbi2023phishy} \\
    \hline
    PhishTank & At PhishTank, the phishing data is reported by users and again tested by others to label them as phishing ones.  & \cite{Shaukat2023},~\cite{Karhani2023206} \\
    \hline
    SMS Spam Collection & Public set of SMS labelled messages, with 5,574 tagged (ham/spam). & \cite{Karhani2023206} \\
    \hline
    The National University of Singapore SMS Corpus & This is a corpus of more than 67,000 SMS messages collected for research at the Department of Computer Science at the National University of Singapore.  & \cite{Karhani2023206} \\
    \hline
    PhishLoad & PhishLoad is a phishing database that contains HTML code, URL and another data relevant to phishing websites. & \cite{Benavides-Astudillo2023} \\
    \hline
    CLAIR collection & Collection of more than 2,500 fraud emails, dating from 1998 and later. & \cite{ALHOGAIL2021102414}\\
    \hline
    IWSPA-AP Corpus & PThe sources of the legitimate email include email
    collections from Wikileaks archives, such as the Democratic National Committee, Hacking Team, Sony emails, Enron Dataset, SpamAssassin, etc. & \cite{8701426} \\
    \hline
    Enron Dataset & It is a dataset that contains about 50,000 spam and 43,000 ham emails. \ac{dl} & \cite{atawneh2023phishing} \\
    \hline
    \end{tabular}
    \caption{Datasets used in the discussed works.}
    \label{tbl:c2:datasets_table}
    \end{table}